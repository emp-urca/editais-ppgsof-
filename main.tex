\documentclass{PPGSOF}

% Esse dois comando a seguir servem para criar o modelo de cabeçalho confome
% customizaçao no modelo ppgsof
\pagestyle{fancy}
\fancyhead[C]{\fancyheadppgsof} 
%----------------------------------------------------------------------------
\def\numedital{009}

\def\subjectedital{Resultado Oficial da avaliação da heteroidentificação e perícia médica para candidatos por ações afirmativas}

\begin{document}

\noindent
\begin{minipage}[t]{0.5\textwidth}
\textbf{Edital nº \numedital/\the\year\ - PPGSOF.}
\end{minipage}%
\begin{minipage}[t]{0.5\textwidth}
\raggedleft Boa Vista, Brasil. \today.
\end{minipage}

\vspace{0.5cm}

\hspace*{8cm}%
\parbox{\dimexpr\textwidth-8.5cm\relax}{%
Edital de \subjectedital ao processo de seleção da turma 2026 no Programa de Pós-Graduação em Sociedade e Fronteiras - PPGSOF.
}

\vspace{0.5cm}

A Comissão de Seleção do Programa de Pós-graduação em Sociedade e Fronteiras - PPGSOF da Universidade Federal de Roraima - UFRR, designada pela Portaria nº 005/2025 – PPGSOF/UFRR, de 6 de outubro de 2025, conforme a Resolução nº 010/2016 – CEPE e com o Regimento Interno do PPGSOF
\vspace{0.5cm}
\\
RESOLVE: 
\vspace{0.5cm}
\\
Art. 1º. Tornar público \subjectedital para o processo seletivo de ingresso no Curso de Mestrado em Sociedade e Fronteiras da UFRR, referente ao Edital nº 035/2025-PPGSOF de 10 de outubro de 2025.
\\
\begin{table}[h]
\centering
\csvreader[
    separator=semicolon,
    tabular=ccc,
    table head=\hline Inscrição & Categoria AF & Resultado \\ \hline,
    late after line=\\\hline
]{dados.csv}{}%
{\csvcoli & \csvcolii & \csvcoliii}
\end{table}



\tikzset{pgfornamentstyle/.style={draw = brown}}
\begin{tikzpicture}[every node/.append style={inner sep=0},remember picture,overlay]

\node at ([shift={(0,-9)}]current page.center) {\scalebox{-1}[1]{} {\begin{tabular}{c}
Evânio Mascarenhas Paulo \\
Presidente Interino
\end{tabular}}};

\node at ([shift={(6,-10)}]current page.center) {\scalebox{-1}[1]{} {\begin{tabular}{c}
Mariana Perreira Cunha \\
Membro
\end{tabular}}};

\node at ([shift={(-6,-10)}]current page.center) {\scalebox{-1}[1]{} {\begin{tabular}{c}
Max André de Araújo \\
Membro
\end{tabular}}};

\end{tikzpicture}

\end{document}